% Created 2018-05-17 Thu 11:06
% Intended LaTeX compiler: pdflatex
\documentclass[11pt]{report}
\usepackage[utf8]{inputenc}
\usepackage[T1]{fontenc}
\usepackage{graphicx}
\usepackage{grffile}
\usepackage{longtable}
\usepackage{wrapfig}
\usepackage{rotating}
\usepackage[normalem]{ulem}
\usepackage{amsmath}
\usepackage{textcomp}
\usepackage{amssymb}
\usepackage{capt-of}
\usepackage{hyperref}
\usepackage[style=chicago-authordate,hyperref=false,backref=false,maxcitenames=3,url=false,isbn=false,eprint=false,doi=true,backend=biber,natbib=true] {biblatex}
\addbibresource{~/Dropbox/current_projects/RadicalOA-2018/Kelty-RP-OA.bib}
\usepackage{setspace}
\usepackage{endnotes}
\usepackage[top=1in,bottom=1.5in,hmargin=1.25in]{geometry}
\author{Christopher Kelty}
\date{\today}
\title{}
\hypersetup{
 pdfauthor={Christopher Kelty},
 pdftitle={},
 pdfkeywords={},
 pdfsubject={},
 pdfcreator={Emacs 25.3.1 (Org mode 9.1.13)}, 
 pdflang={English}}
\begin{document}



\part*{Recursive Publics and Open Access (Version 2)}
\label{sec:orgcbf3389}

\maketitle
\onehalfspacing 
\let\footnote=\endnote
\tolerance=2000
\newlength{\normalparindent}
\setlength{\normalparindent}{\parindent}
\raggedright
\setlength{\parindent}{\normalparindent}

Ten years ago, I published a book with Duke University Press called \emph{Two Bits: The Cultural Significance of Free Software} (\cite{kelty2008twobits}).\footnote{\url{https://twobits.net/download/index.html}} The Press, and my editor Ken Wissoker, were enthusiastically accommodating of my demands to make the book freely and openly available.  They also played along with my desire to release the 'source code' of the book (i.e. HTML files of the chapters with minimal markup), and to compare the data on readers of the open version to print customers.  It was a moment of exploration for both scholarly presses and for me. 

At the time, few authors were doing this other than Yochai Benkler (\cite{benkler2007wealnetw}) and Cory Doctorow\footnote{\url{https://craphound.com/}} both of whom are activists and advocates for free software and open access (OA), much as I have been.  We all shared, I think, a certain fanaticism of the convert that came from recognizing free software as an historically new, and radically different mode of organizing economic and political activity. \emph{Two Bits} gave me a way to talk not only about free software, but about OA and the politics of the university (\cite{kelty2008anthincirc,kelty2014beyoncopytech}). 

Ten years later, I admit to a certain pessimism, if not outright depression at the way things have turned out.  The  promise of free software has foundered, though it has not disappeared, and the question of what it means to achieve OA---or the goals of OA---has been swamped by concerns about costs, arcane details of repositories and versioning, and ritual offerings to the metrics God.  

When I wrote \emph{Two Bits}, it was obvious to me that the collectives who built free software were \emph{essential} to the very structure and operation of a standardized, singular Internet. Today, free software and 'open source' refer to quite dramatically different constellations of practice and people. Free software gathers around itself those committed to the original and purist sense of a shared, collective, process of making software, tools, infrastructures, and hardware that cannot be appropriated by others, cannot be closed or otherwise sealed from view by anyone.  In political terms, I have always identified free software with a very specific, updated, version of classical Millian liberalism.  It sustains a belief in the capacity for collective action and rational thought as aids to establishing a flourishing human livelihood.  Yet it also preserves an outdated blind faith in the automatic functioning of meritorious speech,  that the best ideas will inevitably rise to the top, regardless of context, situation, or history.  It is an updated classical liberalism that saw in software and networks a new place to resist the tyranny of the conventional and the taken for granted.  

By contrast, open source has come to mean something quite different: an ecosystem controlled by an oligopoly of firms which maintains a shared pool of components and frameworks that lower the costs of education, training, and software creation in the service of establishing winner-take-all platforms.  Winner-take-all platforms are built on open source, but they do not carry the principles of freedom or openness all the way through to the platforms themselves.\footnote{For example, Platform Cooperativism \url{https://platform.coop/directory}}.  What open source has become is now almost the opposite of free software---it is authoritarian, plutocratic, and nepotistic, everything liberalism wanted to resist.

For example,precarious labor and platforms such as Uber or Task Rabbit are built upon and rely on the fruits of the labor of 'open source', but the platforms that result do not follow the same principles---they are not open or free in any meaningful sense.  Quite the reverse: their profitability demands that the labor of open source creators is more often directed towards improvements in the underlying infrastructure which serve these platforms---and not the interests of the programmers, engineers, designers, or users who must maintain them, or live by them on a daily basis--- to say nothing of the Uber drivers or task rabbits who live by the platforms.  

Does OA face the same problem?  In part my desire to 'free the source' of my book grew out of the unfinished business of digitizing the scholarly record.  It is an irony that much of the work that went into designing the internet at its outset in the 1980s, such as gopher, WAIS, and the HTML of CERN, was conducted in the name of the digital transformation of the library.  But by 2007, these aims were swamped by attempts to transform the Internet into a giant factory of data extraction, and wars over winner-take-all platforms for spying and surveillance in the name of advertising dollars.  Even in 2006-7 it was clear that this unfinished business of digitizing the scholarly record was going to become a problem---both because it was being overshadowed by other concerns, and because of the danger it would eventually be subjected to the very platformization underway in other realms. 

Because if the platform capitalism of today has ended up being parasitic on the free software tools and infrastructures that enabled it, then why would this not also be true of scholarship more generally?  Are we not witnessing a transition to a world where scholarship is directed---in their very content and organization---towards the profitability of the platforms that ostensibly serve it?\footnote{See for example the figure from 'Rent Seeking by Elsevier,' by Alejandro Posada and George Chen  (\url{http://knowledgegap.org/index.php/sub-projects/rent-seeking-and-financialization-of-the-academic-publishing-industry/preliminary-findings/})}  Is it not possible, that the platforms created to 'serve science' ---Elsevier's increasing acquisition of tools to control the entire life-cycle of research, or ResearchGate's ambition to become the single source for all academics to network and share research---that these platforms might actually end up warping the very content of scholarly production in the service of their profitability? 

To put this even more clearly: OA has come to exist and scholarship is more available and more widely distributed than ever before.  But, scholars now have less control, and have taken less responsibility for the means of production of scientific research, its circulation, and perhaps even the content of that science.  

\chapter*{The Method of Modulation}
\label{sec:org276bfcc}

When I wrote \emph{Two Bits} I organized the argument around the idea of \emph{modulation}: free software is simply one assemblage of technologies, practices, and people aimed at resolving certain problems regarding the  relationship between knowledge (or software tools related to knowledge) and power (\cite{hacking2004histontol,rabinow2003anthtoday}).  Free software as such was and still is changing as each of its elements evolve or are recombined.  Because OA derives some of its practies directly from free software, it is possible to observe how these different elements have in been taken up and worked over in the recent past, as well as how new and surprising elements are combined with OA to transform it.  Looking back on the elements I identified as central to Free Software, one can ask: how is OA different, and what new elements are modulating it into something possibly unrecognizable?  

\section*{\textbf{sharing source code}:}
\label{sec:org68e48c0}
Shareable source code was a concrete and necessary achievement for Free Software to be possible.  Similarly, the necessary ability to circulate digital texts is a significant achievement---but such texts are shareable in a much different way.  For source code, computable streams of text are everything---anything else is a 'blob' like an image, a video or any binary file. But scholarly texts are, with rare exceptions, blobs: Word or Portable Document Format (PDF) files.   What's more, while software programmers may love 'source code', academics generally hate it---anything less than the final, typeset version of a text is considered unfinished (see e.g. the endless disputes over 'author's final versions' plagueing OA).\footnote{see Sherpa/Romeo \url{http://www.sherpa.ac.uk/romeo/index.php}}  Finality is important.  Modifiability of a text, especially in the humanities and social sciences, is acceptable only when it is an experiment of some kind.

In an sense, the source code of science is not a code at all, but a more abstract set of relations between concepts, theories, tools, methods, and the disciplines and networks of people who operate with them, critique them, extend them and try to maintain control over them even as they are shared within these communities. 

\section*{\textbf{defining openness}:}
\label{sec:orga53ad7d}

For Free Software to make sense as a solution, those involved first had to characterize the problem it solved---and they did so by identifying a pathology in the worlds of corporate capitalism and engineering in the 1980s: that computer corporations were closed organizations who re-invented basic tools and infrastructures in a race to dominate a market.    An 'open system,' by contrast, would avoid the waste of 'reinventing the wheel' and of pathological competition, allowing instead  modular, reusable parts that could be modified and recombined to build better things in an upward spiral of innovation.  The 1980s ideas of modularity, modifiability, abstraction barriers, interchangeable units have been essential to the creation of digital infrastructures. 

To propose an 'open science' thus modulates this definition---and the idea works in some sciences better than others.  Aside from the obviously different commercial contexts, philosophers and literary theorists just don't think about openness this way--- theories and arguments may be used as building blocks, but they are not modular in quite the same way.  Only the free circulation of the work, whether for recombination or for reference and critique, remains a \emph{sine qua non} of the theory of openness proposed there.   It is opposed to a system where it is explicit that only certain people have access to the texts (whether that be through limitations of secrecy, or limitations on intellectual property, or an implicit elitism). 

\section*{\textbf{writing and using copyright licenses}:}
\label{sec:org86fe590}

Of all the components of free software that I analyzed, this is the one practice that remains the least transformed--- OA texts use the same CC licenses pioneered in 2001, which were a direct descendant of free software licenses.

A novel modulation of these licenses is the \textbf{OA policies} pioneered in other ways and places as part of the development of OA (the embrace of OA in Brazil for instance, or the spread of OA Policies starting with Harvard and the University of California, and extending to the EU Mandate from 2008 forward).  Today the ability to control the circulation of a text with IP rights is far less economically central to the strategies of publishers than it was in 2007, even if they persist in attempting to do so.  At the same time, funders, states, and universities have all adopted patchwork policies intended to both sustain green OA, and push publishers to innovate their own business models in gold and hybrid OA.  While 'Green OA' is a significant success on paper, the actual use of it to circulate work pales in comparison to the commercial control of circulation on the one hand, and the increasing success of shadow libraries on the other. Repositories have sprung up in every shape and form, but they remain largely \emph{ad hoc}, poorly coordinated, and underfunded solutions to the problem of OA.

\section*{\textbf{coordinating collaborations}:}
\label{sec:orgc143f04}

The \emph{collective} activity of Free Software is ultimately the most significant of its achievements---marrying a form of intensive small-scale interaction amongst programmers, with sophisticated software for managing complex objects (version control and GitHub-like sites).  There has been constant innovation in these tools for controlling, measuring, testing, and maintaining software.

By contrast, the collective activity of scholarship is still largely a pre-modern affair.  It is coordinated largely by the idea of 'writing an article together' and not by working to maintain some larger map of what a research topic, community, or discipline has explored--- what has worked and what has not. 

This focus on the coordination of collaboration seemed to me to be one of the key advantages of free software, but it has turned out to be almost totally absent from the practice or discussion of OA.  Collaboration and the recombination of elements of scholarly practice obviously happens, but it does not depend on OA in any systematic way: there is only the counterfactual that without it, many different kinds of people are excluded from collaboration or even simple participation in, scholarship, something that most active scholars are willfully ignorant of.

\section*{\textbf{fomenting a movement}:}
\label{sec:org6daedfe}

I demoted the idea of a social movement to merely one component of the success of free software, rather than let it be---as most social scientists would have it---the principal container for free software.  As important as movements and their participants are, they are not the whole story.

Is there an OA movement?  Yes and no.  Librarians remain the most activist and organized.  The handful of academics who care about it have shifted to caring about it in primarily a bureaucratic sense, forsaking the cross-organizational aspects of a movement in favor of activism within universities (to which I plead guilty).  But this transformation forsakes the need for addressing the collective, collaborative responsibility for scholarship in favor of letting individual academics, departments, and disciplines be the focus for such debates. 

By contrast, the publishing industry works with a phantasmatic idea of both an OA 'movement' and of the actual practices of scholarship--- they too defer, in speech if not in practice, to the academics themselves, but at the same time must create tools, innovate processes, establish procedures, acquire tools and companies an so on in an effort to capture these phantasms and to prevent academics from collectively doing so on their own. 


And what new components?  The five above were central to free software as I analyzed it up to about 2006.  But OA has other components that are arguably more important to its organization and transformation.

\section*{\textbf{money, i.e. library budgets}:}
\label{sec:org34a524b}
Central to almost all of the politics and debates about OA is the political economy of publication.  From the 'bundles' debates of the 1990s to the gold/green debates of the 2010s, the sole source of money for publication long ago shifted into the library budget. The relationship that library budgets have to other parts of the political economy of research (funding for research itself, debates about tenured/non-tenured, adjunct and other temporary salary structures) has shifted as a result of the demand for OA, leading libraries to re-conceptualize themselves as potential publishers, and publishers to re-conceptualize themselves as serving a 'life cycles' or 'pipeline' of research, not just its dissemination.

\section*{\textbf{metrics}:}
\label{sec:org86adc1a}
More than anything, OA is promoted as a way to continue to feed the metric gods.  OA means more citations, more easily computable data, and more visible uses and re-uses of publications (as well as 'open data' itself, when conceived of as product and not measure).  The innovations in the world of metrics---from the quiet expansion of the platforms of the publishers, to the invention of 'alt metrics' to the enthusiasm of 'open science' for metrics-driven scientific methods---forms a core feature of what 'OA' is today, in a way that was not true of Free Software before it where metrics concerning users, downloads, commits, or lines of code were always after-the-fact measures of quality, and not constitutive ones.  

Other components of this sort might be proposed in an historical-ontological analysis of the transformation of OA, but the main point of an exercise like this is to resist the temptation to clutch OA as if it were the beating heart of a social transformation in science, as if it were a  \emph{thing} that must exist, rather than a configuration of elements at a moment in time.  OA was a solution---but it is too easy to lose sight of the problem. 

\chapter*{Open Access without Recursive Publics}
\label{sec:org3264462}

When we no longer have any commons, but only platforms, will we still have knowledge as we know it?  This is a question at the heart of research in the philosophy and sociology of knowledge---not just a concern for activism or social movements.  If knowledge is socially produced and maintained, then the nature of the social bond surely matters to the nature of that knowledge.   This is not so different than asking whether we  will still have labor or work, as we have long known it, in an age of precarity?  What is the knowledge equivalent of precarity (i.e. not just the existence of precarious knowledge workers, but a kind of \emph{precarious knowledge} as such)?  

Do we not already see the evidence of this in the 'post-truth' of fake news, or the deliberate refusal by those in power to countenance evidence, truth, or established systems of argument and debate? The relationship between knowledge and power is shifting dramatically, because the costs---and the stakes---of producing high quality, authoritative knowledge have also shifted.  It is not so powerful any longer; science does not speak truth to power because truth is no longer so obviously important to power.

Although this is a pessimistic portrait, it may also be a sign of something yet to come.  Free software as a community, has been and still sometimes is critiqued as being an exclusionary space of white male sociality (\cite{nafus_patches_2012,massanari2016fapp,ford2017canedit,reagle2013freeassexis}).  I think this critique is true, but it is less a problem of identity than it is a pathology of a certain form of liberalism: a form that demands that merit consists only in the \emph{content} of the things we say (whether in a political argument, a scientific paper, or a piece of code), and not in the ways we say them, or who is encouraged to say them and who is encouraged to remain silent (\cite{dunbar-hester_low_2014}).

One might, as a result, choose to throw out liberalism altogether as a broken philosophy of governance and liberation.  But it might also be an opportunity to focus much more specifically on a particular problem of liberalism, one that the discourse of OA also relies on to a large extent.  Perhaps it is not the case that merit derives solely from the content of utterances freely and openly circulated, but also from the \emph{ways in which they are uttered, and the dignity of the people who utter them}.  An OA (or a free software) that embraced that principle would demand that we pay attention to different problems:  how are our platforms, infrastructures, tools organized and built to support not just the circulation of putatively true statements, but the ability to say them in situated and particular ways, with respect for the dignity of who is saying them, and with the freedom to explore the limits of \emph{that} kind of liberalism, should we be so lucky to achieve it. 

\theendnotes
 \printbibliography
\end{document}